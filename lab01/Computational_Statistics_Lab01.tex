\documentclass[]{article}
\usepackage{lmodern}
\usepackage{amssymb,amsmath}
\usepackage{ifxetex,ifluatex}
\usepackage{fixltx2e} % provides \textsubscript
\ifnum 0\ifxetex 1\fi\ifluatex 1\fi=0 % if pdftex
  \usepackage[T1]{fontenc}
  \usepackage[utf8]{inputenc}
\else % if luatex or xelatex
  \ifxetex
    \usepackage{mathspec}
  \else
    \usepackage{fontspec}
  \fi
  \defaultfontfeatures{Ligatures=TeX,Scale=MatchLowercase}
\fi
% use upquote if available, for straight quotes in verbatim environments
\IfFileExists{upquote.sty}{\usepackage{upquote}}{}
% use microtype if available
\IfFileExists{microtype.sty}{%
\usepackage{microtype}
\UseMicrotypeSet[protrusion]{basicmath} % disable protrusion for tt fonts
}{}
\usepackage[margin=1in]{geometry}
\usepackage{hyperref}
\hypersetup{unicode=true,
            pdftitle={Computational Statistics - Lab 01},
            pdfauthor={Annalena Erhard (anner218) and Maximilian Pfundstein (maxpf364)},
            pdfborder={0 0 0},
            breaklinks=true}
\urlstyle{same}  % don't use monospace font for urls
\usepackage{color}
\usepackage{fancyvrb}
\newcommand{\VerbBar}{|}
\newcommand{\VERB}{\Verb[commandchars=\\\{\}]}
\DefineVerbatimEnvironment{Highlighting}{Verbatim}{commandchars=\\\{\}}
% Add ',fontsize=\small' for more characters per line
\usepackage{framed}
\definecolor{shadecolor}{RGB}{248,248,248}
\newenvironment{Shaded}{\begin{snugshade}}{\end{snugshade}}
\newcommand{\KeywordTok}[1]{\textcolor[rgb]{0.13,0.29,0.53}{\textbf{#1}}}
\newcommand{\DataTypeTok}[1]{\textcolor[rgb]{0.13,0.29,0.53}{#1}}
\newcommand{\DecValTok}[1]{\textcolor[rgb]{0.00,0.00,0.81}{#1}}
\newcommand{\BaseNTok}[1]{\textcolor[rgb]{0.00,0.00,0.81}{#1}}
\newcommand{\FloatTok}[1]{\textcolor[rgb]{0.00,0.00,0.81}{#1}}
\newcommand{\ConstantTok}[1]{\textcolor[rgb]{0.00,0.00,0.00}{#1}}
\newcommand{\CharTok}[1]{\textcolor[rgb]{0.31,0.60,0.02}{#1}}
\newcommand{\SpecialCharTok}[1]{\textcolor[rgb]{0.00,0.00,0.00}{#1}}
\newcommand{\StringTok}[1]{\textcolor[rgb]{0.31,0.60,0.02}{#1}}
\newcommand{\VerbatimStringTok}[1]{\textcolor[rgb]{0.31,0.60,0.02}{#1}}
\newcommand{\SpecialStringTok}[1]{\textcolor[rgb]{0.31,0.60,0.02}{#1}}
\newcommand{\ImportTok}[1]{#1}
\newcommand{\CommentTok}[1]{\textcolor[rgb]{0.56,0.35,0.01}{\textit{#1}}}
\newcommand{\DocumentationTok}[1]{\textcolor[rgb]{0.56,0.35,0.01}{\textbf{\textit{#1}}}}
\newcommand{\AnnotationTok}[1]{\textcolor[rgb]{0.56,0.35,0.01}{\textbf{\textit{#1}}}}
\newcommand{\CommentVarTok}[1]{\textcolor[rgb]{0.56,0.35,0.01}{\textbf{\textit{#1}}}}
\newcommand{\OtherTok}[1]{\textcolor[rgb]{0.56,0.35,0.01}{#1}}
\newcommand{\FunctionTok}[1]{\textcolor[rgb]{0.00,0.00,0.00}{#1}}
\newcommand{\VariableTok}[1]{\textcolor[rgb]{0.00,0.00,0.00}{#1}}
\newcommand{\ControlFlowTok}[1]{\textcolor[rgb]{0.13,0.29,0.53}{\textbf{#1}}}
\newcommand{\OperatorTok}[1]{\textcolor[rgb]{0.81,0.36,0.00}{\textbf{#1}}}
\newcommand{\BuiltInTok}[1]{#1}
\newcommand{\ExtensionTok}[1]{#1}
\newcommand{\PreprocessorTok}[1]{\textcolor[rgb]{0.56,0.35,0.01}{\textit{#1}}}
\newcommand{\AttributeTok}[1]{\textcolor[rgb]{0.77,0.63,0.00}{#1}}
\newcommand{\RegionMarkerTok}[1]{#1}
\newcommand{\InformationTok}[1]{\textcolor[rgb]{0.56,0.35,0.01}{\textbf{\textit{#1}}}}
\newcommand{\WarningTok}[1]{\textcolor[rgb]{0.56,0.35,0.01}{\textbf{\textit{#1}}}}
\newcommand{\AlertTok}[1]{\textcolor[rgb]{0.94,0.16,0.16}{#1}}
\newcommand{\ErrorTok}[1]{\textcolor[rgb]{0.64,0.00,0.00}{\textbf{#1}}}
\newcommand{\NormalTok}[1]{#1}
\usepackage{longtable,booktabs}
\usepackage{graphicx,grffile}
\makeatletter
\def\maxwidth{\ifdim\Gin@nat@width>\linewidth\linewidth\else\Gin@nat@width\fi}
\def\maxheight{\ifdim\Gin@nat@height>\textheight\textheight\else\Gin@nat@height\fi}
\makeatother
% Scale images if necessary, so that they will not overflow the page
% margins by default, and it is still possible to overwrite the defaults
% using explicit options in \includegraphics[width, height, ...]{}
\setkeys{Gin}{width=\maxwidth,height=\maxheight,keepaspectratio}
\IfFileExists{parskip.sty}{%
\usepackage{parskip}
}{% else
\setlength{\parindent}{0pt}
\setlength{\parskip}{6pt plus 2pt minus 1pt}
}
\setlength{\emergencystretch}{3em}  % prevent overfull lines
\providecommand{\tightlist}{%
  \setlength{\itemsep}{0pt}\setlength{\parskip}{0pt}}
\setcounter{secnumdepth}{5}
% Redefines (sub)paragraphs to behave more like sections
\ifx\paragraph\undefined\else
\let\oldparagraph\paragraph
\renewcommand{\paragraph}[1]{\oldparagraph{#1}\mbox{}}
\fi
\ifx\subparagraph\undefined\else
\let\oldsubparagraph\subparagraph
\renewcommand{\subparagraph}[1]{\oldsubparagraph{#1}\mbox{}}
\fi

%%% Use protect on footnotes to avoid problems with footnotes in titles
\let\rmarkdownfootnote\footnote%
\def\footnote{\protect\rmarkdownfootnote}

%%% Change title format to be more compact
\usepackage{titling}

% Create subtitle command for use in maketitle
\newcommand{\subtitle}[1]{
  \posttitle{
    \begin{center}\large#1\end{center}
    }
}

\setlength{\droptitle}{-2em}

  \title{Computational Statistics - Lab 01}
    \pretitle{\vspace{\droptitle}\centering\huge}
  \posttitle{\par}
    \author{Annalena Erhard (anner218) and Maximilian Pfundstein (maxpf364)}
    \preauthor{\centering\large\emph}
  \postauthor{\par}
      \predate{\centering\large\emph}
  \postdate{\par}
    \date{2019-02-03}


\begin{document}
\maketitle

{
\setcounter{tocdepth}{3}
\tableofcontents
}
\section{Question 1: Be Careful When
Comparing}\label{question-1-be-careful-when-comparing}

\begin{Shaded}
\begin{Highlighting}[]
\NormalTok{################################################################################}
\CommentTok{# Question 1 - Be Careful When Comparing}
\NormalTok{################################################################################}

\NormalTok{x1 =}\StringTok{ }\DecValTok{1}\OperatorTok{/}\DecValTok{3}
\NormalTok{x2 =}\StringTok{ }\DecValTok{1}\OperatorTok{/}\DecValTok{4}

\ControlFlowTok{if}\NormalTok{ (x1}\OperatorTok{-}\NormalTok{x2 }\OperatorTok{==}\StringTok{ }\DecValTok{1}\OperatorTok{/}\DecValTok{12}\NormalTok{) \{}
  \KeywordTok{print}\NormalTok{(}\StringTok{"Substraction is correct."}\NormalTok{)}
\NormalTok{\} }\ControlFlowTok{else}\NormalTok{ \{}
  \KeywordTok{print}\NormalTok{(}\StringTok{"Substraction is wrong."}\NormalTok{)}
\NormalTok{\}}
\end{Highlighting}
\end{Shaded}

\begin{verbatim}
## [1] "Substraction is wrong."
\end{verbatim}

\begin{Shaded}
\begin{Highlighting}[]
\NormalTok{x1 =}\StringTok{ }\DecValTok{1}
\NormalTok{x2 =}\StringTok{ }\DecValTok{1}\OperatorTok{/}\DecValTok{2}

\ControlFlowTok{if}\NormalTok{ (x1}\OperatorTok{-}\NormalTok{x2 }\OperatorTok{==}\StringTok{ }\DecValTok{1}\OperatorTok{/}\DecValTok{2}\NormalTok{) \{}
  \KeywordTok{print}\NormalTok{(}\StringTok{"Substraction is correct."}\NormalTok{)}
\NormalTok{\} }\ControlFlowTok{else}\NormalTok{ \{}
  \KeywordTok{print}\NormalTok{(}\StringTok{"Substraction is wrong."}\NormalTok{)}
\NormalTok{\}}
\end{Highlighting}
\end{Shaded}

\begin{verbatim}
## [1] "Substraction is correct."
\end{verbatim}

\textbf{Questions:}

\begin{enumerate}
\def\labelenumi{\arabic{enumi}.}
\tightlist
\item
  Check the results of the snippets. Comment what is going on.
\item
  If there are any problems, suggest improvements.
\end{enumerate}

\textbf{Answers:}

\begin{enumerate}
\def\labelenumi{\arabic{enumi}.}
\item
  The first substraction is not wrong - it is perfectly working as
  defined in \href{https://en.wikipedia.org/wiki/IEEE_754}{IEEE\_754}
  which is a commonly used definition for floating point numbers. To
  make a long strory short: You have an infinite amount of real numbers
  for instance between \(0.0\) and \(1.0\) but just 32 or 64 bits for
  the represetation (so \(2^{32}\) states or \(2^{64}\) states) so it's
  impossible to represent every number (t.ex. try writing down 1/3 in
  the decimal system, at one point you simply must stop). The second
  substraction does not have a floating point error as you can represent
  multiples of the power of two in the binary system \(2^{-1} = 0.5\).
\item
  You can either just use more bits for representing the numbers until
  you have the desired precision or use/write a class which can handle a
  specific amount of numbers behind the decimal point (which will be
  slower for sure). Or it can be handled by the following code snippet:
\end{enumerate}

\begin{Shaded}
\begin{Highlighting}[]
\NormalTok{x1 =}\StringTok{ }\DecValTok{1}\OperatorTok{/}\DecValTok{3}
\NormalTok{x2 =}\StringTok{ }\DecValTok{1}\OperatorTok{/}\DecValTok{4}

\ControlFlowTok{if}\NormalTok{ (}\KeywordTok{all.equal}\NormalTok{(x1}\OperatorTok{-}\NormalTok{x2, }\DecValTok{1}\OperatorTok{/}\DecValTok{12}\NormalTok{)) \{}
  \KeywordTok{print}\NormalTok{(}\StringTok{"Substraction is correct."}\NormalTok{)}
\NormalTok{\} }\ControlFlowTok{else}\NormalTok{ \{}
  \KeywordTok{print}\NormalTok{(}\StringTok{"Substraction is wrong."}\NormalTok{)}
\NormalTok{\}}
\end{Highlighting}
\end{Shaded}

\begin{verbatim}
## [1] "Substraction is correct."
\end{verbatim}

\section{Question 2: Derivative}\label{question-2-derivative}

\textbf{Question:} Write your own R function to calculate the derivative
of \texttt{f(x)\ =\ x} in this way with \(\epsilon = 10^{-15}\).

\begin{Shaded}
\begin{Highlighting}[]
\NormalTok{################################################################################}
\CommentTok{# Question 2: Derivative}
\NormalTok{################################################################################}

\NormalTok{epsilon =}\StringTok{ }\DecValTok{10}\OperatorTok{^}\NormalTok{(}\OperatorTok{-}\DecValTok{15}\NormalTok{)}

\NormalTok{f_prime =}\StringTok{ }\ControlFlowTok{function}\NormalTok{(x) \{}
  \KeywordTok{return}\NormalTok{( (}\KeywordTok{f}\NormalTok{(x }\OperatorTok{+}\StringTok{ }\NormalTok{epsilon) }\OperatorTok{-}\StringTok{ }\KeywordTok{f}\NormalTok{(x)) }\OperatorTok{/}\StringTok{ }\NormalTok{epsilon)}
\NormalTok{\}}

\NormalTok{f =}\StringTok{ }\ControlFlowTok{function}\NormalTok{(x) \{}
  \KeywordTok{return}\NormalTok{(x)}
\NormalTok{\}}
\end{Highlighting}
\end{Shaded}

\textbf{Question:} Evaluate your derivative function at \texttt{x\ =\ 1}
and \texttt{x\ =\ 100000}.

\begin{Shaded}
\begin{Highlighting}[]
\NormalTok{first =}\StringTok{ }\KeywordTok{f_prime}\NormalTok{(}\DecValTok{1}\NormalTok{)}
\NormalTok{second =}\StringTok{ }\KeywordTok{f_prime}\NormalTok{(}\DecValTok{100000}\NormalTok{)}

\KeywordTok{print}\NormalTok{(first)}
\end{Highlighting}
\end{Shaded}

\begin{verbatim}
## [1] 1.110223
\end{verbatim}

\begin{Shaded}
\begin{Highlighting}[]
\KeywordTok{print}\NormalTok{(second)}
\end{Highlighting}
\end{Shaded}

\begin{verbatim}
## [1] 0
\end{verbatim}

The following plots show the function \texttt{f} and \texttt{f\_prime}
in the interval \texttt{0...20}. We observe that the derivative seems to
take discrete values and is \texttt{0} around \texttt{x\ =\ 16}.

\includegraphics{Computational_Statistics_Lab01_files/figure-latex/unnamed-chunk-6-1.pdf}

\textbf{Question:} What values did you obtain? What are the true values?
Explain the reasons behind the discovered differences.

\textbf{Answer:} The values and plots can be seen above.

The derivate of \texttt{f(x)\ =\ x} is
\texttt{f\textquotesingle{}(x)\ =\ 1} so the true slope is \texttt{1} at
all spots.

As \texttt{epsilon} is a really small number and we do calculations with
a rather big number (\texttt{x}) we run into precision problems which
are more heavy if the magnitudes of the numbers is greatly different.
Therefore we see that if we take \texttt{x\ =\ 1} the error is smaller,
but still big enough to give an undesired result. As we take
\texttt{x\ =\ 100000} the difference in magnitude increased further so
we obtain the weird result \texttt{0} which s obviously totally wrong.

If you evaluate only the nominator you will see that it's \texttt{0}
after \texttt{x\ =\ 16} so the result will always be \texttt{0}
(assuming the denominator is unequal to \texttt{0}).

\section{Question 3: Variance}\label{question-3-variance}

\textbf{Question:} Write your own R function, \texttt{myvar}, to
estimate the variance in this way.

\begin{Shaded}
\begin{Highlighting}[]
\NormalTok{################################################################################}
\CommentTok{# Question 3: Variance}
\NormalTok{################################################################################}

\NormalTok{myvar =}\StringTok{ }\ControlFlowTok{function}\NormalTok{(x) }\KeywordTok{return}\NormalTok{(}\DecValTok{1}\OperatorTok{/}\NormalTok{(}\KeywordTok{length}\NormalTok{(x)}\OperatorTok{-}\DecValTok{1}\NormalTok{)  }\OperatorTok{*}\StringTok{ }\NormalTok{(}\KeywordTok{sum}\NormalTok{(x}\OperatorTok{^}\DecValTok{2}\NormalTok{) }\OperatorTok{-}\StringTok{ }\NormalTok{(}\KeywordTok{sum}\NormalTok{(x)}\OperatorTok{^}\DecValTok{2}\NormalTok{)}\OperatorTok{/}\KeywordTok{length}\NormalTok{(x)))}
\end{Highlighting}
\end{Shaded}

\textbf{Question:} Generate a vector \(x = (x_1, ..., x_{10000})\) with
10000 random numbers with mean \(10^8\) and variance \(1\).

\begin{Shaded}
\begin{Highlighting}[]
\NormalTok{v =}\StringTok{ }\KeywordTok{rnorm}\NormalTok{(}\DecValTok{10000}\NormalTok{, }\DataTypeTok{mean =} \DecValTok{10}\OperatorTok{^}\DecValTok{8}\NormalTok{, }\DataTypeTok{sd =} \DecValTok{1}\NormalTok{)}
\end{Highlighting}
\end{Shaded}

\textbf{Question:} For each subset
\(X_i = \{x1,...,x_i \}, i = 1, ..., 10000\) compute the difference
\(Y_i = myvar(X_i)-var(X_i)\), where \(var(X_i)\) is the standard
variance estimation function in R. Plot the dependence \(Y_i\) on \(i\).
Draw conclusions from this plot. How well does your function work? Can
you explain the behaviour?

\includegraphics{Computational_Statistics_Lab01_files/figure-latex/unnamed-chunk-9-1.pdf}

\textbf{Answer:} When interpretating the plot one has to keep in mind
that each subset contains one value more than the previous one, so we
should observe that we're getting better estimations for the variance
which an increased index. Let's focus on the first values
(\texttt{index\ \textless{}\ 10}). They're highly scattered which is to
be expected for such a small number of data points. With increasing
\texttt{index} we see that some pattern is repeated. We have datapoints
around \texttt{-1} as well as trails towards the center. If we look at
the trail that is on the upper half, ending at around an index of
\texttt{100} we observe the following values:

\begin{Shaded}
\begin{Highlighting}[]
\KeywordTok{kable}\NormalTok{(X[X}\OperatorTok{$}\NormalTok{index }\OperatorTok{>}\StringTok{ }\DecValTok{70} \OperatorTok{&}\StringTok{ }\NormalTok{X}\OperatorTok{$}\NormalTok{index }\OperatorTok{<}\StringTok{ }\DecValTok{100}\NormalTok{,])}
\end{Highlighting}
\end{Shaded}

\begin{longtable}[]{@{}lrrrr@{}}
\toprule
& index & value & vec\_myvar & vec\_var\tabularnewline
\midrule
\endhead
71 & 71 & 0.9447615 & 1.828571 & 0.8838099\tabularnewline
72 & 72 & 0.8903787 & 1.802817 & 0.9124382\tabularnewline
73 & 73 & -0.9275357 & 0.000000 & 0.9275357\tabularnewline
74 & 74 & 0.8373387 & 1.753425 & 0.9160860\tabularnewline
75 & 75 & 0.8247103 & 1.729730 & 0.9050194\tabularnewline
76 & 76 & 0.7957881 & 1.706667 & 0.9108785\tabularnewline
77 & 77 & 0.7514780 & 1.684211 & 0.9327326\tabularnewline
78 & 78 & 0.7289432 & 1.662338 & 0.9333945\tabularnewline
79 & 79 & 0.6988023 & 1.641026 & 0.9422234\tabularnewline
80 & 80 & -0.9321492 & 0.000000 & 0.9321492\tabularnewline
81 & 81 & 0.6780125 & 1.600000 & 0.9219875\tabularnewline
82 & 82 & 0.6419519 & 1.580247 & 0.9382950\tabularnewline
83 & 83 & -0.9276504 & 0.000000 & 0.9276504\tabularnewline
84 & 84 & -0.9169604 & 0.000000 & 0.9169604\tabularnewline
85 & 85 & -0.9267945 & 0.000000 & 0.9267945\tabularnewline
86 & 86 & 0.5863524 & 1.505882 & 0.9195299\tabularnewline
87 & 87 & 0.5794368 & 1.488372 & 0.9089353\tabularnewline
88 & 88 & 0.5716958 & 1.471264 & 0.8995686\tabularnewline
89 & 89 & -0.8912002 & 0.000000 & 0.8912002\tabularnewline
90 & 90 & -0.8821134 & 0.000000 & 0.8821134\tabularnewline
91 & 91 & -0.8774752 & 0.000000 & 0.8774752\tabularnewline
92 & 92 & -0.8679647 & 0.000000 & 0.8679647\tabularnewline
93 & 93 & -0.8656059 & 0.000000 & 0.8656059\tabularnewline
94 & 94 & 0.5053668 & 1.376344 & 0.8709773\tabularnewline
95 & 95 & 1.8529564 & 2.723404 & 0.8704478\tabularnewline
96 & 96 & -2.2100560 & -1.347368 & 0.8626875\tabularnewline
97 & 97 & -2.2141970 & -1.333333 & 0.8808637\tabularnewline
98 & 98 & -0.8916640 & 0.000000 & 0.8916640\tabularnewline
99 & 99 & 3.0348994 & 3.918367 & 0.8834679\tabularnewline
\bottomrule
\end{longtable}

As we see the trails are \textbf{not} continuous as they heavily
fluctuade. So, as it seems like they do not converge and are not
continuous, this function (\texttt{my\_var()}) seems not to be a good
estimator for the variance. As we have rather big values with just a
small variance from the created data and we have the sum of quite a few
data, we might run into an underflow and precision problems.

\textbf{Question:} How can you better implement a variance estimator?
Find and implement a formula that will give the same results as
\texttt{var()}?

\textbf{Answer:} We use this formula:

\[s = \frac{\sum_{i=1}^{n}(x_i-\overline{x})^2}{n-1}\]

This is the sample standard deviance and thus is biased (if we assume
our \(\vec{v}\) as a sample from a population).

\begin{Shaded}
\begin{Highlighting}[]
\NormalTok{custom_variance =}\StringTok{ }\ControlFlowTok{function}\NormalTok{(x) \{}
\NormalTok{  diff_mean =}\StringTok{ }\NormalTok{x }\OperatorTok{-}\StringTok{ }\KeywordTok{mean}\NormalTok{(x)}
  \KeywordTok{return}\NormalTok{(}\KeywordTok{sum}\NormalTok{(diff_mean}\OperatorTok{^}\DecValTok{2} \OperatorTok{/}\StringTok{ }\NormalTok{(}\KeywordTok{length}\NormalTok{(x) }\OperatorTok{-}\StringTok{ }\DecValTok{1}\NormalTok{)))}
\NormalTok{\}}
\end{Highlighting}
\end{Shaded}

\includegraphics{Computational_Statistics_Lab01_files/figure-latex/unnamed-chunk-12-1.pdf}

It can be seen that this function does a way better job at converging
and returns almost (or exactly) the same values as the built in function
and thus covering the \texttt{var()} plot almost perfectly.

\section{Question 4: Linear Algebra}\label{question-4-linear-algebra}

\textbf{Question:} Import the data set to R.

\begin{Shaded}
\begin{Highlighting}[]
\NormalTok{################################################################################}
\CommentTok{# Question 4: Linear Algebra}
\NormalTok{################################################################################}

\NormalTok{data =}\StringTok{ }\KeywordTok{read.csv2}\NormalTok{(}\StringTok{"tecator.csv"}\NormalTok{, }\DataTypeTok{sep=}\StringTok{","}\NormalTok{, }\DataTypeTok{dec=}\StringTok{"."}\NormalTok{)}
\KeywordTok{kable}\NormalTok{(}\KeywordTok{head}\NormalTok{(data[, }\KeywordTok{c}\NormalTok{(}\DecValTok{1}\NormalTok{, }\DecValTok{101}\NormalTok{, }\DecValTok{102}\NormalTok{, }\DecValTok{103}\NormalTok{, }\DecValTok{104}\NormalTok{)]))}
\end{Highlighting}
\end{Shaded}

\begin{longtable}[]{@{}rrrrr@{}}
\toprule
Sample & Channel100 & Fat & Protein & Moisture\tabularnewline
\midrule
\endhead
1 & 2.81920 & 22.5 & 16.7 & 60.5\tabularnewline
2 & 3.17942 & 40.1 & 13.5 & 46.0\tabularnewline
3 & 2.54816 & 8.4 & 20.5 & 71.0\tabularnewline
4 & 2.79622 & 5.9 & 20.7 & 72.8\tabularnewline
5 & 3.13753 & 25.5 & 15.5 & 58.3\tabularnewline
6 & 3.45307 & 42.7 & 13.7 & 44.0\tabularnewline
\bottomrule
\end{longtable}

\textbf{Question:} Optimal regression coeffcients can be found by
solving a system of the type \(A\vec{\beta} = \vec{b}\) where
\(A = X^TX\) and \(\vec{b} = X^T\vec{y}\). Compute \(A\) and \(\vec{b}\)
for the given data set. The matrix \(X\) are the observations of the
absorbance records, levels of moisture and fat, while \(\vec{y}\) are
the protein levels.

\begin{Shaded}
\begin{Highlighting}[]
\NormalTok{X =}\StringTok{ }\KeywordTok{as.matrix}\NormalTok{(data[, }\KeywordTok{c}\NormalTok{(}\DecValTok{1}\OperatorTok{:}\DecValTok{102}\NormalTok{, }\DecValTok{104}\NormalTok{)])}
\NormalTok{Y =}\StringTok{ }\KeywordTok{as.matrix}\NormalTok{(data[, }\KeywordTok{c}\NormalTok{(}\DecValTok{103}\NormalTok{)])}
\NormalTok{A =}\StringTok{ }\KeywordTok{t}\NormalTok{(X) }\OperatorTok\StringTok{ }\NormalTok{X}
\NormalTok{b =}\StringTok{ }\KeywordTok{t}\NormalTok{(X) }\OperatorTok\StringTok{ }\NormalTok{Y}
\end{Highlighting}
\end{Shaded}

\textbf{Question:} Try to solve \(A\vec{\beta} = \vec{b}\) with default
solver \texttt{solve()}. What kind of result did you get? How can this
result be explained?

\begin{Shaded}
\begin{Highlighting}[]
\KeywordTok{tryCatch}\NormalTok{(}
    \DataTypeTok{expr =}\NormalTok{ \{}
\NormalTok{        beta =}\StringTok{ }\KeywordTok{solve}\NormalTok{(A) }\OperatorTok\StringTok{ }\NormalTok{b}
\NormalTok{    \},}
    \DataTypeTok{error =} \ControlFlowTok{function}\NormalTok{(e)\{ }
        \KeywordTok{paste}\NormalTok{(}\StringTok{"That escalated rather quickly: "}\NormalTok{, e)}
\NormalTok{    \}}
\NormalTok{)}
\end{Highlighting}
\end{Shaded}

\begin{verbatim}
## [1] "That escalated rather quickly:  Error in solve.default(A): System ist für den Rechner singulär: reziproke Konditionszahl = 3.02468e-17\n"
\end{verbatim}

\textbf{Question:} Check the condition number of the matrix A (function
\texttt{kappa()}) and consider how it is related to your conclusion in
step 3.

\begin{Shaded}
\begin{Highlighting}[]
\KeywordTok{kappa}\NormalTok{(A)}
\end{Highlighting}
\end{Shaded}

\begin{verbatim}
## [1] 4.274694e+15
\end{verbatim}

As we mostly work with \emph{rational} numbers we are used to the fact
that almost every number has a inverse. An inverse \(a^{-1}\) is defined
as that element that, multiplied with \(a\) results in the \emph{neutral
element}, at least for multiplications in arithmetics. In arithmetics we
only have one number that does not have an multiplicative inverse which
is \texttt{0} as \texttt{1/0} is undefined.

With matirces there are way more such matrices that do not have an
inverse, exactly then when:

\begin{itemize}
\tightlist
\item
  The matrix is not a square.
\item
  The determinant and thus the span is \texttt{0}.
\end{itemize}

If we imagine a matrix as a linear transformation the determinant is the
factor by which the space is streched or compressed. Thus a determinant
of \texttt{0} tells us if the given linear transformation is squishing
the amount of dimensions.

So it's reasonable that we have matrices that do not have an inverse. If
we wanted to use the inverse for calculating or solving an equation, we
have to find a different way (t.ex. QR-decomposition).

The \texttt{kappa()} function computes an estimate of the condition
number of a matrix. Given a linear equation \(Ax = b\) the number gives
us an estimation of how inaccurate the approximation of \(x\) is going
to be. One can also say that it says how much \(x\) is going to change
in respect to \(b\). So if we have a large condition number it means
that a small error in \(b\) is likely to cause a large error in \(x\).
As our number here is rather large, we can conclude that our features
are linearly dependant.

\textbf{Question:} Scale the data set and repeat steps 2-4. How has the
result changed and why?

\begin{Shaded}
\begin{Highlighting}[]
\NormalTok{data_scaled =}\StringTok{ }\KeywordTok{scale}\NormalTok{(data)}

\NormalTok{X =}\StringTok{ }\KeywordTok{as.matrix}\NormalTok{(data_scaled[, }\KeywordTok{c}\NormalTok{(}\DecValTok{1}\OperatorTok{:}\DecValTok{102}\NormalTok{, }\DecValTok{104}\NormalTok{)])}
\NormalTok{Y =}\StringTok{ }\KeywordTok{as.matrix}\NormalTok{(data_scaled[, }\KeywordTok{c}\NormalTok{(}\DecValTok{103}\NormalTok{)])}
\NormalTok{A =}\StringTok{ }\KeywordTok{t}\NormalTok{(X) }\OperatorTok\StringTok{ }\NormalTok{X}
\NormalTok{b =}\StringTok{ }\KeywordTok{t}\NormalTok{(X) }\OperatorTok\StringTok{ }\NormalTok{Y}

\NormalTok{beta =}\StringTok{ }\KeywordTok{solve}\NormalTok{(A) }\OperatorTok\StringTok{ }\NormalTok{b}
\end{Highlighting}
\end{Shaded}

The result has changed as we scaled the data. Before we ran into
computational issues as the scale for each feature was on a different
scope/scale which can lead to those errors. The \texttt{sclae()} made
them of equal size and thus solved the problem.

Last but not least, let's look at the new conditional number, which
should be smaller:

\begin{Shaded}
\begin{Highlighting}[]
\KeywordTok{kappa}\NormalTok{(A)}
\end{Highlighting}
\end{Shaded}

\begin{verbatim}
## [1] 664318664630
\end{verbatim}

It is smaller as we'd have expected. So everyone is happy and we can
conclude this lab!

\begin{Shaded}
\begin{Highlighting}[]
\NormalTok{                                      _.}\OperatorTok{--}\StringTok{"""--,}
\StringTok{                                    .'          `\textbackslash{}}
\StringTok{  .-""""""-.                      .'              |}
\StringTok{ /          '.                   /            .-._/}
\StringTok{|             `.                |             |}
\StringTok{ \textbackslash{}              \textbackslash{}          .-._ |          _   \textbackslash{}}
\StringTok{  `""'-.         \textbackslash{}_.-.     \textbackslash{}   `          ( \textbackslash{}__/}
\StringTok{        |             )     '=.       .,   \textbackslash{}  }
\StringTok{       /             (         \textbackslash{}     /  \textbackslash{}  /}
\StringTok{     /`               `\textbackslash{}        |   /    `'}
\StringTok{     '..-`\textbackslash{}        _.-. `\textbackslash{} _.__/   .=.}
\StringTok{          |  _    / \textbackslash{}  '.-`    `-.'  /}
\StringTok{          \textbackslash{}_/ |  |   './ _     _  \textbackslash{}.'}
\StringTok{               '-'    | /       \textbackslash{} |  }
\StringTok{                      |  .-. .-.  |   H A V E   A   N I C E   D A Y !}
\StringTok{                      \textbackslash{} / o| |o \textbackslash{} /}
\StringTok{                       |   / \textbackslash{}   |           M O O !}
\StringTok{                      / `"`}\DataTypeTok{   }\StringTok{`"` \textbackslash{}}
\StringTok{                     /             \textbackslash{}}
\StringTok{                    | '._.'         \textbackslash{}}
\StringTok{                    |  /             |}
\StringTok{                     \textbackslash{} |             |}
\StringTok{                      ||    _    _   /}
\StringTok{                      /|\textbackslash{}  (_\textbackslash{}  /_) /}
\StringTok{                      \textbackslash{} }\CharTok{\textbackslash{}'}\StringTok{._  ` '_.'}
\StringTok{                       `""` `"""`}
\end{Highlighting}
\end{Shaded}

\section{Source Code}\label{source-code}

\begin{Shaded}
\begin{Highlighting}[]
\NormalTok{knitr}\OperatorTok{::}\NormalTok{opts_chunk}\OperatorTok{$}\KeywordTok{set}\NormalTok{(}\DataTypeTok{echo =} \OtherTok{TRUE}\NormalTok{, }\DataTypeTok{cache =} \OtherTok{FALSE}\NormalTok{, }\DataTypeTok{include =} \OtherTok{TRUE}\NormalTok{, }\DataTypeTok{eval =} \OtherTok{TRUE}\NormalTok{)}
\KeywordTok{library}\NormalTok{(ggplot2)}
\KeywordTok{library}\NormalTok{(knitr)}
\KeywordTok{library}\NormalTok{(gridExtra)}

\NormalTok{################################################################################}
\CommentTok{# Question 1 - Be Careful When Comparing}
\NormalTok{################################################################################}

\NormalTok{x1 =}\StringTok{ }\DecValTok{1}\OperatorTok{/}\DecValTok{3}
\NormalTok{x2 =}\StringTok{ }\DecValTok{1}\OperatorTok{/}\DecValTok{4}

\ControlFlowTok{if}\NormalTok{ (x1}\OperatorTok{-}\NormalTok{x2 }\OperatorTok{==}\StringTok{ }\DecValTok{1}\OperatorTok{/}\DecValTok{12}\NormalTok{) \{}
  \KeywordTok{print}\NormalTok{(}\StringTok{"Substraction is correct."}\NormalTok{)}
\NormalTok{\} }\ControlFlowTok{else}\NormalTok{ \{}
  \KeywordTok{print}\NormalTok{(}\StringTok{"Substraction is wrong."}\NormalTok{)}
\NormalTok{\}}


\NormalTok{x1 =}\StringTok{ }\DecValTok{1}
\NormalTok{x2 =}\StringTok{ }\DecValTok{1}\OperatorTok{/}\DecValTok{2}

\ControlFlowTok{if}\NormalTok{ (x1}\OperatorTok{-}\NormalTok{x2 }\OperatorTok{==}\StringTok{ }\DecValTok{1}\OperatorTok{/}\DecValTok{2}\NormalTok{) \{}
  \KeywordTok{print}\NormalTok{(}\StringTok{"Substraction is correct."}\NormalTok{)}
\NormalTok{\} }\ControlFlowTok{else}\NormalTok{ \{}
  \KeywordTok{print}\NormalTok{(}\StringTok{"Substraction is wrong."}\NormalTok{)}
\NormalTok{\}}

\NormalTok{x1 =}\StringTok{ }\DecValTok{1}\OperatorTok{/}\DecValTok{3}
\NormalTok{x2 =}\StringTok{ }\DecValTok{1}\OperatorTok{/}\DecValTok{4}

\ControlFlowTok{if}\NormalTok{ (}\KeywordTok{all.equal}\NormalTok{(x1}\OperatorTok{-}\NormalTok{x2, }\DecValTok{1}\OperatorTok{/}\DecValTok{12}\NormalTok{)) \{}
  \KeywordTok{print}\NormalTok{(}\StringTok{"Substraction is correct."}\NormalTok{)}
\NormalTok{\} }\ControlFlowTok{else}\NormalTok{ \{}
  \KeywordTok{print}\NormalTok{(}\StringTok{"Substraction is wrong."}\NormalTok{)}
\NormalTok{\}}


\NormalTok{################################################################################}
\CommentTok{# Question 2: Derivative}
\NormalTok{################################################################################}

\NormalTok{epsilon =}\StringTok{ }\DecValTok{10}\OperatorTok{^}\NormalTok{(}\OperatorTok{-}\DecValTok{15}\NormalTok{)}

\NormalTok{f_prime =}\StringTok{ }\ControlFlowTok{function}\NormalTok{(x) \{}
  \KeywordTok{return}\NormalTok{( (}\KeywordTok{f}\NormalTok{(x }\OperatorTok{+}\StringTok{ }\NormalTok{epsilon) }\OperatorTok{-}\StringTok{ }\KeywordTok{f}\NormalTok{(x)) }\OperatorTok{/}\StringTok{ }\NormalTok{epsilon)}
\NormalTok{\}}

\NormalTok{f =}\StringTok{ }\ControlFlowTok{function}\NormalTok{(x) \{}
  \KeywordTok{return}\NormalTok{(x)}
\NormalTok{\}}


\NormalTok{first =}\StringTok{ }\KeywordTok{f_prime}\NormalTok{(}\DecValTok{1}\NormalTok{)}
\NormalTok{second =}\StringTok{ }\KeywordTok{f_prime}\NormalTok{(}\DecValTok{100000}\NormalTok{)}

\KeywordTok{print}\NormalTok{(first)}
\KeywordTok{print}\NormalTok{(second)}


\NormalTok{sequence =}\StringTok{ }\KeywordTok{seq}\NormalTok{(}\DataTypeTok{from =} \DecValTok{0}\NormalTok{, }\DataTypeTok{to =} \DecValTok{20}\NormalTok{, }\DataTypeTok{by =} \DecValTok{1}\NormalTok{)}
\NormalTok{func =}\StringTok{ }\KeywordTok{f}\NormalTok{(sequence)}
\NormalTok{deri =}\StringTok{ }\KeywordTok{f_prime}\NormalTok{(sequence)}
\NormalTok{df =}\StringTok{ }\KeywordTok{data.frame}\NormalTok{(sequence, deri)}

\NormalTok{p1 =}\StringTok{ }\KeywordTok{ggplot}\NormalTok{(df) }\OperatorTok{+}
\StringTok{  }\KeywordTok{geom_line}\NormalTok{(}\KeywordTok{aes}\NormalTok{(}\DataTypeTok{x =}\NormalTok{ sequence, }\DataTypeTok{y =}\NormalTok{ func), }\DataTypeTok{color =} \StringTok{"#C70039"}\NormalTok{) }\OperatorTok{+}
\StringTok{  }\KeywordTok{labs}\NormalTok{(}\DataTypeTok{title =} \StringTok{"f_prime(x)"}\NormalTok{, }\DataTypeTok{y =} \StringTok{"f_prime(x)"}\NormalTok{, }\DataTypeTok{x =} \StringTok{"x"}\NormalTok{) }\OperatorTok{+}
\StringTok{  }\KeywordTok{theme_minimal}\NormalTok{()}

\NormalTok{p2 =}\StringTok{ }\KeywordTok{ggplot}\NormalTok{(df) }\OperatorTok{+}
\StringTok{  }\KeywordTok{geom_line}\NormalTok{(}\KeywordTok{aes}\NormalTok{(}\DataTypeTok{x =}\NormalTok{ sequence, }\DataTypeTok{y =}\NormalTok{ deri), }\DataTypeTok{color =} \StringTok{"#900C3F"}\NormalTok{) }\OperatorTok{+}
\StringTok{  }\KeywordTok{labs}\NormalTok{(}\DataTypeTok{title =} \StringTok{"f(x)"}\NormalTok{, }\DataTypeTok{y =} \StringTok{"f(x)"}\NormalTok{, }\DataTypeTok{x =} \StringTok{"x"}\NormalTok{) }\OperatorTok{+}
\StringTok{  }\KeywordTok{theme_minimal}\NormalTok{()}

\KeywordTok{grid.arrange}\NormalTok{(p1, p2, }\DataTypeTok{nrow =} \DecValTok{1}\NormalTok{)}


\NormalTok{################################################################################}
\CommentTok{# Question 3: Variance}
\NormalTok{################################################################################}

\NormalTok{myvar =}\StringTok{ }\ControlFlowTok{function}\NormalTok{(x) }\KeywordTok{return}\NormalTok{(}\DecValTok{1}\OperatorTok{/}\NormalTok{(}\KeywordTok{length}\NormalTok{(x)}\OperatorTok{-}\DecValTok{1}\NormalTok{)  }\OperatorTok{*}\StringTok{ }\NormalTok{(}\KeywordTok{sum}\NormalTok{(x}\OperatorTok{^}\DecValTok{2}\NormalTok{) }\OperatorTok{-}\StringTok{ }\NormalTok{(}\KeywordTok{sum}\NormalTok{(x)}\OperatorTok{^}\DecValTok{2}\NormalTok{)}\OperatorTok{/}\KeywordTok{length}\NormalTok{(x)))}


\NormalTok{v =}\StringTok{ }\KeywordTok{rnorm}\NormalTok{(}\DecValTok{10000}\NormalTok{, }\DataTypeTok{mean =} \DecValTok{10}\OperatorTok{^}\DecValTok{8}\NormalTok{, }\DataTypeTok{sd =} \DecValTok{1}\NormalTok{)}


\NormalTok{X =}\StringTok{ }\KeywordTok{data.frame}\NormalTok{()}

\ControlFlowTok{for}\NormalTok{ (i }\ControlFlowTok{in} \DecValTok{1}\OperatorTok{:}\KeywordTok{length}\NormalTok{(v)) \{}
\NormalTok{  Xi =}\StringTok{ }\NormalTok{v[}\DecValTok{1}\OperatorTok{:}\NormalTok{i]}
\NormalTok{  vec_myvar =}\StringTok{ }\KeywordTok{myvar}\NormalTok{(}\KeywordTok{as.vector}\NormalTok{(Xi))}
\NormalTok{  vec_var =}\StringTok{ }\KeywordTok{var}\NormalTok{(Xi)}
\NormalTok{  Yi =}\StringTok{ }\NormalTok{vec_myvar }\OperatorTok{-}\StringTok{ }\NormalTok{vec_var}
\NormalTok{  Yi_index =}\StringTok{ }\KeywordTok{list}\NormalTok{(}\DataTypeTok{index =}\NormalTok{ i, }\DataTypeTok{value =}\NormalTok{ Yi, }\DataTypeTok{vec_myvar =}\NormalTok{ vec_myvar, }\DataTypeTok{vec_var =}\NormalTok{ vec_var)}
\NormalTok{  X =}\StringTok{ }\KeywordTok{rbind}\NormalTok{(X, Yi_index)}
\NormalTok{\}}

\KeywordTok{ggplot}\NormalTok{(X[}\DecValTok{2}\OperatorTok{:}\KeywordTok{nrow}\NormalTok{(X),]) }\OperatorTok{+}
\StringTok{  }\KeywordTok{geom_point}\NormalTok{(}\KeywordTok{aes}\NormalTok{(}\DataTypeTok{x =}\NormalTok{ index, }\DataTypeTok{y =}\NormalTok{ value, }\DataTypeTok{colour =} \StringTok{"Difference"}\NormalTok{)) }\OperatorTok{+}
\StringTok{  }\KeywordTok{geom_point}\NormalTok{(}\KeywordTok{aes}\NormalTok{(}\DataTypeTok{x =}\NormalTok{ index, }\DataTypeTok{y =}\NormalTok{ vec_myvar,  }\DataTypeTok{colour =} \StringTok{"my_var()"}\NormalTok{)) }\OperatorTok{+}
\StringTok{  }\KeywordTok{geom_point}\NormalTok{(}\KeywordTok{aes}\NormalTok{(}\DataTypeTok{x =}\NormalTok{ index, }\DataTypeTok{y =}\NormalTok{ vec_var, }\DataTypeTok{colour =} \StringTok{"var()"}\NormalTok{)) }\OperatorTok{+}
\StringTok{  }\KeywordTok{labs}\NormalTok{(}\DataTypeTok{title =} \StringTok{"Difference in Variance"}\NormalTok{, }\DataTypeTok{y =} \StringTok{"Variance"}\NormalTok{,}
  \DataTypeTok{x =} \StringTok{"Sequence"}\NormalTok{, }\DataTypeTok{color =} \StringTok{"Legend"}\NormalTok{) }\OperatorTok{+}
\StringTok{  }\KeywordTok{scale_color_manual}\NormalTok{(}\DataTypeTok{values =} \KeywordTok{c}\NormalTok{(}\StringTok{"#C70039"}\NormalTok{, }\StringTok{"#407AFF"}\NormalTok{, }\StringTok{"#FFC300"}\NormalTok{)) }\OperatorTok{+}
\StringTok{  }\KeywordTok{scale_x_log10}\NormalTok{() }\OperatorTok{+}\StringTok{ }
\StringTok{  }\KeywordTok{theme_minimal}\NormalTok{()}


\KeywordTok{kable}\NormalTok{(X[X}\OperatorTok{$}\NormalTok{index }\OperatorTok{>}\StringTok{ }\DecValTok{70} \OperatorTok{&}\StringTok{ }\NormalTok{X}\OperatorTok{$}\NormalTok{index }\OperatorTok{<}\StringTok{ }\DecValTok{100}\NormalTok{,])}


\NormalTok{custom_variance =}\StringTok{ }\ControlFlowTok{function}\NormalTok{(x) \{}
\NormalTok{  diff_mean =}\StringTok{ }\NormalTok{x }\OperatorTok{-}\StringTok{ }\KeywordTok{mean}\NormalTok{(x)}
  \KeywordTok{return}\NormalTok{(}\KeywordTok{sum}\NormalTok{(diff_mean}\OperatorTok{^}\DecValTok{2} \OperatorTok{/}\StringTok{ }\NormalTok{(}\KeywordTok{length}\NormalTok{(x) }\OperatorTok{-}\StringTok{ }\DecValTok{1}\NormalTok{)))}
\NormalTok{\}}


\ControlFlowTok{for}\NormalTok{ (i }\ControlFlowTok{in} \DecValTok{1}\OperatorTok{:}\KeywordTok{length}\NormalTok{(v)) \{}
\NormalTok{  Xi =}\StringTok{ }\NormalTok{v[}\DecValTok{1}\OperatorTok{:}\NormalTok{i]}
\NormalTok{  X}\OperatorTok{$}\NormalTok{vec_customvar[[i]] =}\StringTok{ }\KeywordTok{custom_variance}\NormalTok{(}\KeywordTok{as.vector}\NormalTok{(Xi))}
\NormalTok{\}}

\KeywordTok{ggplot}\NormalTok{(X[}\DecValTok{2}\OperatorTok{:}\KeywordTok{nrow}\NormalTok{(X),]) }\OperatorTok{+}
\StringTok{  }\KeywordTok{geom_point}\NormalTok{(}\KeywordTok{aes}\NormalTok{(}\DataTypeTok{x =}\NormalTok{ index, }\DataTypeTok{y =}\NormalTok{ value, }\DataTypeTok{colour =} \StringTok{"Difference"}\NormalTok{)) }\OperatorTok{+}
\StringTok{  }\KeywordTok{geom_point}\NormalTok{(}\KeywordTok{aes}\NormalTok{(}\DataTypeTok{x =}\NormalTok{ index, }\DataTypeTok{y =}\NormalTok{ vec_myvar,  }\DataTypeTok{colour =} \StringTok{"my_var()"}\NormalTok{)) }\OperatorTok{+}
\StringTok{  }\KeywordTok{geom_point}\NormalTok{(}\KeywordTok{aes}\NormalTok{(}\DataTypeTok{x =}\NormalTok{ index, }\DataTypeTok{y =}\NormalTok{ vec_var, }\DataTypeTok{colour =} \StringTok{"var()"}\NormalTok{)) }\OperatorTok{+}
\StringTok{  }\KeywordTok{geom_point}\NormalTok{(}\KeywordTok{aes}\NormalTok{(}\DataTypeTok{x =}\NormalTok{ index, }\DataTypeTok{y =}\NormalTok{ vec_customvar, }\DataTypeTok{colour =} \StringTok{"custom_variance()"}\NormalTok{)) }\OperatorTok{+}
\StringTok{  }\KeywordTok{labs}\NormalTok{(}\DataTypeTok{title =} \StringTok{"Difference in Variance"}\NormalTok{, }\DataTypeTok{y =} \StringTok{"Variance"}\NormalTok{,}
  \DataTypeTok{x =} \StringTok{"Sequence"}\NormalTok{, }\DataTypeTok{color =} \StringTok{"Legend"}\NormalTok{) }\OperatorTok{+}
\StringTok{  }\KeywordTok{scale_color_manual}\NormalTok{(}\DataTypeTok{values =} \KeywordTok{c}\NormalTok{(}\StringTok{"#17202A"}\NormalTok{, }\StringTok{"#C70039"}\NormalTok{, }\StringTok{"#407AFF"}\NormalTok{, }\StringTok{"#FFC300"}\NormalTok{)) }\OperatorTok{+}
\StringTok{  }\KeywordTok{scale_x_log10}\NormalTok{() }\OperatorTok{+}\StringTok{ }
\StringTok{  }\KeywordTok{theme_minimal}\NormalTok{()}



\NormalTok{################################################################################}
\CommentTok{# Question 4: Linear Algebra}
\NormalTok{################################################################################}

\NormalTok{data =}\StringTok{ }\KeywordTok{read.csv2}\NormalTok{(}\StringTok{"tecator.csv"}\NormalTok{, }\DataTypeTok{sep=}\StringTok{","}\NormalTok{, }\DataTypeTok{dec=}\StringTok{"."}\NormalTok{)}
\KeywordTok{kable}\NormalTok{(}\KeywordTok{head}\NormalTok{(data[, }\KeywordTok{c}\NormalTok{(}\DecValTok{1}\NormalTok{, }\DecValTok{101}\NormalTok{, }\DecValTok{102}\NormalTok{, }\DecValTok{103}\NormalTok{, }\DecValTok{104}\NormalTok{)]))}


\NormalTok{X =}\StringTok{ }\KeywordTok{as.matrix}\NormalTok{(data[, }\KeywordTok{c}\NormalTok{(}\DecValTok{1}\OperatorTok{:}\DecValTok{102}\NormalTok{, }\DecValTok{104}\NormalTok{)])}
\NormalTok{Y =}\StringTok{ }\KeywordTok{as.matrix}\NormalTok{(data[, }\KeywordTok{c}\NormalTok{(}\DecValTok{103}\NormalTok{)])}
\NormalTok{A =}\StringTok{ }\KeywordTok{t}\NormalTok{(X) }\OperatorTok\StringTok{ }\NormalTok{X}
\NormalTok{b =}\StringTok{ }\KeywordTok{t}\NormalTok{(X) }\OperatorTok\StringTok{ }\NormalTok{Y}


\KeywordTok{tryCatch}\NormalTok{(}
    \DataTypeTok{expr =}\NormalTok{ \{}
\NormalTok{        beta =}\StringTok{ }\KeywordTok{solve}\NormalTok{(A) }\OperatorTok\StringTok{ }\NormalTok{b}
\NormalTok{    \},}
    \DataTypeTok{error =} \ControlFlowTok{function}\NormalTok{(e)\{ }
        \KeywordTok{paste}\NormalTok{(}\StringTok{"That escalated rather quickly: "}\NormalTok{, e)}
\NormalTok{    \}}
\NormalTok{)}


\KeywordTok{kappa}\NormalTok{(A)}


\NormalTok{data_scaled =}\StringTok{ }\KeywordTok{scale}\NormalTok{(data)}

\NormalTok{X =}\StringTok{ }\KeywordTok{as.matrix}\NormalTok{(data_scaled[, }\KeywordTok{c}\NormalTok{(}\DecValTok{1}\OperatorTok{:}\DecValTok{102}\NormalTok{, }\DecValTok{104}\NormalTok{)])}
\NormalTok{Y =}\StringTok{ }\KeywordTok{as.matrix}\NormalTok{(data_scaled[, }\KeywordTok{c}\NormalTok{(}\DecValTok{103}\NormalTok{)])}
\NormalTok{A =}\StringTok{ }\KeywordTok{t}\NormalTok{(X) }\OperatorTok\StringTok{ }\NormalTok{X}
\NormalTok{b =}\StringTok{ }\KeywordTok{t}\NormalTok{(X) }\OperatorTok\StringTok{ }\NormalTok{Y}

\NormalTok{beta =}\StringTok{ }\KeywordTok{solve}\NormalTok{(A) }\OperatorTok\StringTok{ }\NormalTok{b}


\KeywordTok{kappa}\NormalTok{(A)}

\NormalTok{                                      _.}\OperatorTok{--}\StringTok{"""--,}
\StringTok{                                    .'          `\textbackslash{}}
\StringTok{  .-""""""-.                      .'              |}
\StringTok{ /          '.                   /            .-._/}
\StringTok{|             `.                |             |}
\StringTok{ \textbackslash{}              \textbackslash{}          .-._ |          _   \textbackslash{}}
\StringTok{  `""'-.         \textbackslash{}_.-.     \textbackslash{}   `          ( \textbackslash{}__/}
\StringTok{        |             )     '=.       .,   \textbackslash{}  }
\StringTok{       /             (         \textbackslash{}     /  \textbackslash{}  /}
\StringTok{     /`               `\textbackslash{}        |   /    `'}
\StringTok{     '..-`\textbackslash{}        _.-. `\textbackslash{} _.__/   .=.}
\StringTok{          |  _    / \textbackslash{}  '.-`    `-.'  /}
\StringTok{          \textbackslash{}_/ |  |   './ _     _  \textbackslash{}.'}
\StringTok{               '-'    | /       \textbackslash{} |  }
\StringTok{                      |  .-. .-.  |   H A V E   A   N I C E   D A Y !}
\StringTok{                      \textbackslash{} / o| |o \textbackslash{} /}
\StringTok{                       |   / \textbackslash{}   |           M O O !}
\StringTok{                      / `"`}\DataTypeTok{   }\StringTok{`"` \textbackslash{}}
\StringTok{                     /             \textbackslash{}}
\StringTok{                    | '._.'         \textbackslash{}}
\StringTok{                    |  /             |}
\StringTok{                     \textbackslash{} |             |}
\StringTok{                      ||    _    _   /}
\StringTok{                      /|\textbackslash{}  (_\textbackslash{}  /_) /}
\StringTok{                      \textbackslash{} }\CharTok{\textbackslash{}'}\StringTok{._  ` '_.'}
\StringTok{                       `""` `"""`}
\end{Highlighting}
\end{Shaded}


\end{document}
